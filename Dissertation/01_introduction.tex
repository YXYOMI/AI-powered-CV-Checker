\chapter{Introduction}
% \addcontentsline{toc}{chapter}{Introduction}
\label{ch:introduction}



\section{Description \& Motivation}
% existing CV checker. figure out how to achieve
% NLP techniques 


% main characters in resume. NER
% skills in job description and resume, matching

Following the responses to the COVID-19 crisis, the labour market has been affected and unemployment has sustained risen \cite{mayhew2020covid}. According to \cite{yang2019decision}, the employment situation of graduates has become serious. As the first contact between companies and recruiters \cite{thoms1999resume}, resumes are regarded as significant tools that can enhance the attraction of job seekers to the companies \cite{hornsby1995resume} and some characteristics of the resume can have a crucial impact on the selection decisions from the organization to the recruiter \cite{van2015difference}. A job description is a useful tool to illustrate the task, duties and responsibilities of a position \cite{enwiki:1043689877}, which is a significant document that a job seeker should look at before applying for a position. Through the job description, job seekers can self-check their fitness between themselves and the particular position, and modify their resumes to match the position better. 

When people aim to apply for many positions, especially in different areas, it is vital to check the resumes and the fitness between the resumes and the job descriptions. According to \cite{bhatt2016resume}, manually checking resumes is very time-consuming. In this case, the AI-powered CV checker is proposed to help job seekers consummate their resumes. 

The key modules of the resume checking system are checking the information in the resume automatically and finding the matching degree between the resume and the position, which are the two main subjects in this project.

\section{Aims \& Objectives}
% aims and objectives

NLP (Natural Language Processing) techniques and machine learning methods have the ability to parse and extract useful information from unstructured data such as resumes \cite{sinha2021resume}. The project aims to create an AI-powered CV checker based on NLP and machine learning algorithms, which could parse and check the resume, and give general advice on the uploaded resumes. Additionally, the CV checker should compare the resumes and the target job description, return the match rate and the related feedback.
 
The mechanism of the CV checker would be as the combination of \href{https://www.jobscan.co/}{Jobscan} and \href{https://careerset.com/}{CareerSet}. First of all, users can upload their resumes to get started and add the job description of the target job. Then the application will scan and parse the uploaded data, and return the result. The result contains the match rate between the resume and the job description, which might include the title match and the skill match, and also contains the general feedback and recommendation of the resume based on formatting check, wording check, et cetera.
 
The key objectives are:

\begin{enumerate}
    \item Web development.
    \item Parsing the unstructured resume data.
    \item Finding suitable NLP and machine learning techniques to identify key information from resume for resume content checking.
    \item Finding techniques for skill extraction and skill matching from both job description and resume to compare the cv and the position then getting the match rate.
    \item Summarize and visualize the results.
    
\end{enumerate}


\section{Dissertation Structure}

The structure of this dissertation is ordered as follows: 

\begin{enumerate}
    \item Chapter 1 described the overall scene of the project area, defined the research (development) question and gave the general introduction of this project including the aims and key objectives.
    
    \item Chapter 2 will introduce the background and related work in the subject of this project, from the aspects of existing CV checkers, special unstructured data to the related NLP and Machine Learning techniques. The content will include the basic introduction and background of each aspect, with a description of the existing systems, projects or techniques that related to our project implementation.
    
    \item Chapter 3 will illustrate the technologies and tools used in each implementation step of the application, including the technical introduction and the theory or some math basis of each technique.
    
    
    \item Based on the methodologies, the design and implementation details in each part of the project will be demonstrated in Chapter 4, including the overall design of the system, UML diagrams, data preparation and back-end algorithm implementation of each target task.
    
    \item The evaluation of each sub-task and the whole application will be provided in Chapter 5. The experimental results and analysis are the main component of this chapter, where the experimental results would be presented in the form of figures or tables.
    
    \item The last chapter of the dissertation will conclude the project by summarizing the achievement and contributions from the project, taking reflection through the developmental phases and identifying the future work of the project. 
    
\end{enumerate}
